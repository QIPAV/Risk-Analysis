\section{Introduction}
During the start-up phase of a project it is important to do a thorough risk analysis, so that we can be reactive to scenarios that can put our project in danger. The risk analysis identifies all possible hazards that can cause problems for the project development, and determines the probability these. By identifying these scenarios we can minimize or eliminate the different risks that can put the project development in danger. Examples of risk scenarios is budget overrun, injuries, long term sick-leave etc. The scenarios is categorized, with the probability of occurrence and the severity of this scenario. The scenarios is further analysed in an risk matrix which helps us to see which risks are most critical for our project. \\ \\
The risk analysis analyses the consequence of the risks. It describes what assets that are at risk. It finds the source of the hazard or actors behind the risk, and it describes what type of event or incident that is considered. The probability and severity of the risk is defined. 
\\ \\
Source: http://www.jakeman.com.au/media/whats-right-with-risk-matrices